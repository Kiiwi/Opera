
% Default to the notebook output style

    


% Inherit from the specified cell style.




    
\documentclass[11pt]{article}

    
    
    \usepackage[T1]{fontenc}
    % Nicer default font (+ math font) than Computer Modern for most use cases
    \usepackage{mathpazo}

    % Basic figure setup, for now with no caption control since it's done
    % automatically by Pandoc (which extracts ![](path) syntax from Markdown).
    \usepackage{graphicx}
    % We will generate all images so they have a width \maxwidth. This means
    % that they will get their normal width if they fit onto the page, but
    % are scaled down if they would overflow the margins.
    \makeatletter
    \def\maxwidth{\ifdim\Gin@nat@width>\linewidth\linewidth
    \else\Gin@nat@width\fi}
    \makeatother
    \let\Oldincludegraphics\includegraphics
    % Set max figure width to be 80% of text width, for now hardcoded.
    \renewcommand{\includegraphics}[1]{\Oldincludegraphics[width=.8\maxwidth]{#1}}
    % Ensure that by default, figures have no caption (until we provide a
    % proper Figure object with a Caption API and a way to capture that
    % in the conversion process - todo).
    \usepackage{caption}
    \DeclareCaptionLabelFormat{nolabel}{}
    \captionsetup{labelformat=nolabel}

    \usepackage{adjustbox} % Used to constrain images to a maximum size 
    \usepackage{xcolor} % Allow colors to be defined
    \usepackage{enumerate} % Needed for markdown enumerations to work
    \usepackage{geometry} % Used to adjust the document margins
    \usepackage{amsmath} % Equations
    \usepackage{amssymb} % Equations
    \usepackage{textcomp} % defines textquotesingle
    % Hack from http://tex.stackexchange.com/a/47451/13684:
    \AtBeginDocument{%
        \def\PYZsq{\textquotesingle}% Upright quotes in Pygmentized code
    }
    \usepackage{upquote} % Upright quotes for verbatim code
    \usepackage{eurosym} % defines \euro
    \usepackage[mathletters]{ucs} % Extended unicode (utf-8) support
    \usepackage[utf8x]{inputenc} % Allow utf-8 characters in the tex document
    \usepackage{fancyvrb} % verbatim replacement that allows latex
    \usepackage{grffile} % extends the file name processing of package graphics 
                         % to support a larger range 
    % The hyperref package gives us a pdf with properly built
    % internal navigation ('pdf bookmarks' for the table of contents,
    % internal cross-reference links, web links for URLs, etc.)
    \usepackage{hyperref}
    \usepackage{longtable} % longtable support required by pandoc >1.10
    \usepackage{booktabs}  % table support for pandoc > 1.12.2
    \usepackage[inline]{enumitem} % IRkernel/repr support (it uses the enumerate* environment)
    \usepackage[normalem]{ulem} % ulem is needed to support strikethroughs (\sout)
                                % normalem makes italics be italics, not underlines
    

    
    
    % Colors for the hyperref package
    \definecolor{urlcolor}{rgb}{0,.145,.698}
    \definecolor{linkcolor}{rgb}{.71,0.21,0.01}
    \definecolor{citecolor}{rgb}{.12,.54,.11}

    % ANSI colors
    \definecolor{ansi-black}{HTML}{3E424D}
    \definecolor{ansi-black-intense}{HTML}{282C36}
    \definecolor{ansi-red}{HTML}{E75C58}
    \definecolor{ansi-red-intense}{HTML}{B22B31}
    \definecolor{ansi-green}{HTML}{00A250}
    \definecolor{ansi-green-intense}{HTML}{007427}
    \definecolor{ansi-yellow}{HTML}{DDB62B}
    \definecolor{ansi-yellow-intense}{HTML}{B27D12}
    \definecolor{ansi-blue}{HTML}{208FFB}
    \definecolor{ansi-blue-intense}{HTML}{0065CA}
    \definecolor{ansi-magenta}{HTML}{D160C4}
    \definecolor{ansi-magenta-intense}{HTML}{A03196}
    \definecolor{ansi-cyan}{HTML}{60C6C8}
    \definecolor{ansi-cyan-intense}{HTML}{258F8F}
    \definecolor{ansi-white}{HTML}{C5C1B4}
    \definecolor{ansi-white-intense}{HTML}{A1A6B2}

    % commands and environments needed by pandoc snippets
    % extracted from the output of `pandoc -s`
    \providecommand{\tightlist}{%
      \setlength{\itemsep}{0pt}\setlength{\parskip}{0pt}}
    \DefineVerbatimEnvironment{Highlighting}{Verbatim}{commandchars=\\\{\}}
    % Add ',fontsize=\small' for more characters per line
    \newenvironment{Shaded}{}{}
    \newcommand{\KeywordTok}[1]{\textcolor[rgb]{0.00,0.44,0.13}{\textbf{{#1}}}}
    \newcommand{\DataTypeTok}[1]{\textcolor[rgb]{0.56,0.13,0.00}{{#1}}}
    \newcommand{\DecValTok}[1]{\textcolor[rgb]{0.25,0.63,0.44}{{#1}}}
    \newcommand{\BaseNTok}[1]{\textcolor[rgb]{0.25,0.63,0.44}{{#1}}}
    \newcommand{\FloatTok}[1]{\textcolor[rgb]{0.25,0.63,0.44}{{#1}}}
    \newcommand{\CharTok}[1]{\textcolor[rgb]{0.25,0.44,0.63}{{#1}}}
    \newcommand{\StringTok}[1]{\textcolor[rgb]{0.25,0.44,0.63}{{#1}}}
    \newcommand{\CommentTok}[1]{\textcolor[rgb]{0.38,0.63,0.69}{\textit{{#1}}}}
    \newcommand{\OtherTok}[1]{\textcolor[rgb]{0.00,0.44,0.13}{{#1}}}
    \newcommand{\AlertTok}[1]{\textcolor[rgb]{1.00,0.00,0.00}{\textbf{{#1}}}}
    \newcommand{\FunctionTok}[1]{\textcolor[rgb]{0.02,0.16,0.49}{{#1}}}
    \newcommand{\RegionMarkerTok}[1]{{#1}}
    \newcommand{\ErrorTok}[1]{\textcolor[rgb]{1.00,0.00,0.00}{\textbf{{#1}}}}
    \newcommand{\NormalTok}[1]{{#1}}
    
    % Additional commands for more recent versions of Pandoc
    \newcommand{\ConstantTok}[1]{\textcolor[rgb]{0.53,0.00,0.00}{{#1}}}
    \newcommand{\SpecialCharTok}[1]{\textcolor[rgb]{0.25,0.44,0.63}{{#1}}}
    \newcommand{\VerbatimStringTok}[1]{\textcolor[rgb]{0.25,0.44,0.63}{{#1}}}
    \newcommand{\SpecialStringTok}[1]{\textcolor[rgb]{0.73,0.40,0.53}{{#1}}}
    \newcommand{\ImportTok}[1]{{#1}}
    \newcommand{\DocumentationTok}[1]{\textcolor[rgb]{0.73,0.13,0.13}{\textit{{#1}}}}
    \newcommand{\AnnotationTok}[1]{\textcolor[rgb]{0.38,0.63,0.69}{\textbf{\textit{{#1}}}}}
    \newcommand{\CommentVarTok}[1]{\textcolor[rgb]{0.38,0.63,0.69}{\textbf{\textit{{#1}}}}}
    \newcommand{\VariableTok}[1]{\textcolor[rgb]{0.10,0.09,0.49}{{#1}}}
    \newcommand{\ControlFlowTok}[1]{\textcolor[rgb]{0.00,0.44,0.13}{\textbf{{#1}}}}
    \newcommand{\OperatorTok}[1]{\textcolor[rgb]{0.40,0.40,0.40}{{#1}}}
    \newcommand{\BuiltInTok}[1]{{#1}}
    \newcommand{\ExtensionTok}[1]{{#1}}
    \newcommand{\PreprocessorTok}[1]{\textcolor[rgb]{0.74,0.48,0.00}{{#1}}}
    \newcommand{\AttributeTok}[1]{\textcolor[rgb]{0.49,0.56,0.16}{{#1}}}
    \newcommand{\InformationTok}[1]{\textcolor[rgb]{0.38,0.63,0.69}{\textbf{\textit{{#1}}}}}
    \newcommand{\WarningTok}[1]{\textcolor[rgb]{0.38,0.63,0.69}{\textbf{\textit{{#1}}}}}
    
    
    % Define a nice break command that doesn't care if a line doesn't already
    % exist.
    \def\br{\hspace*{\fill} \\* }
    % Math Jax compatability definitions
    \def\gt{>}
    \def\lt{<}
    % Document parameters
    \title{Report}
    
    
    

    % Pygments definitions
    
\makeatletter
\def\PY@reset{\let\PY@it=\relax \let\PY@bf=\relax%
    \let\PY@ul=\relax \let\PY@tc=\relax%
    \let\PY@bc=\relax \let\PY@ff=\relax}
\def\PY@tok#1{\csname PY@tok@#1\endcsname}
\def\PY@toks#1+{\ifx\relax#1\empty\else%
    \PY@tok{#1}\expandafter\PY@toks\fi}
\def\PY@do#1{\PY@bc{\PY@tc{\PY@ul{%
    \PY@it{\PY@bf{\PY@ff{#1}}}}}}}
\def\PY#1#2{\PY@reset\PY@toks#1+\relax+\PY@do{#2}}

\expandafter\def\csname PY@tok@w\endcsname{\def\PY@tc##1{\textcolor[rgb]{0.73,0.73,0.73}{##1}}}
\expandafter\def\csname PY@tok@c\endcsname{\let\PY@it=\textit\def\PY@tc##1{\textcolor[rgb]{0.25,0.50,0.50}{##1}}}
\expandafter\def\csname PY@tok@cp\endcsname{\def\PY@tc##1{\textcolor[rgb]{0.74,0.48,0.00}{##1}}}
\expandafter\def\csname PY@tok@k\endcsname{\let\PY@bf=\textbf\def\PY@tc##1{\textcolor[rgb]{0.00,0.50,0.00}{##1}}}
\expandafter\def\csname PY@tok@kp\endcsname{\def\PY@tc##1{\textcolor[rgb]{0.00,0.50,0.00}{##1}}}
\expandafter\def\csname PY@tok@kt\endcsname{\def\PY@tc##1{\textcolor[rgb]{0.69,0.00,0.25}{##1}}}
\expandafter\def\csname PY@tok@o\endcsname{\def\PY@tc##1{\textcolor[rgb]{0.40,0.40,0.40}{##1}}}
\expandafter\def\csname PY@tok@ow\endcsname{\let\PY@bf=\textbf\def\PY@tc##1{\textcolor[rgb]{0.67,0.13,1.00}{##1}}}
\expandafter\def\csname PY@tok@nb\endcsname{\def\PY@tc##1{\textcolor[rgb]{0.00,0.50,0.00}{##1}}}
\expandafter\def\csname PY@tok@nf\endcsname{\def\PY@tc##1{\textcolor[rgb]{0.00,0.00,1.00}{##1}}}
\expandafter\def\csname PY@tok@nc\endcsname{\let\PY@bf=\textbf\def\PY@tc##1{\textcolor[rgb]{0.00,0.00,1.00}{##1}}}
\expandafter\def\csname PY@tok@nn\endcsname{\let\PY@bf=\textbf\def\PY@tc##1{\textcolor[rgb]{0.00,0.00,1.00}{##1}}}
\expandafter\def\csname PY@tok@ne\endcsname{\let\PY@bf=\textbf\def\PY@tc##1{\textcolor[rgb]{0.82,0.25,0.23}{##1}}}
\expandafter\def\csname PY@tok@nv\endcsname{\def\PY@tc##1{\textcolor[rgb]{0.10,0.09,0.49}{##1}}}
\expandafter\def\csname PY@tok@no\endcsname{\def\PY@tc##1{\textcolor[rgb]{0.53,0.00,0.00}{##1}}}
\expandafter\def\csname PY@tok@nl\endcsname{\def\PY@tc##1{\textcolor[rgb]{0.63,0.63,0.00}{##1}}}
\expandafter\def\csname PY@tok@ni\endcsname{\let\PY@bf=\textbf\def\PY@tc##1{\textcolor[rgb]{0.60,0.60,0.60}{##1}}}
\expandafter\def\csname PY@tok@na\endcsname{\def\PY@tc##1{\textcolor[rgb]{0.49,0.56,0.16}{##1}}}
\expandafter\def\csname PY@tok@nt\endcsname{\let\PY@bf=\textbf\def\PY@tc##1{\textcolor[rgb]{0.00,0.50,0.00}{##1}}}
\expandafter\def\csname PY@tok@nd\endcsname{\def\PY@tc##1{\textcolor[rgb]{0.67,0.13,1.00}{##1}}}
\expandafter\def\csname PY@tok@s\endcsname{\def\PY@tc##1{\textcolor[rgb]{0.73,0.13,0.13}{##1}}}
\expandafter\def\csname PY@tok@sd\endcsname{\let\PY@it=\textit\def\PY@tc##1{\textcolor[rgb]{0.73,0.13,0.13}{##1}}}
\expandafter\def\csname PY@tok@si\endcsname{\let\PY@bf=\textbf\def\PY@tc##1{\textcolor[rgb]{0.73,0.40,0.53}{##1}}}
\expandafter\def\csname PY@tok@se\endcsname{\let\PY@bf=\textbf\def\PY@tc##1{\textcolor[rgb]{0.73,0.40,0.13}{##1}}}
\expandafter\def\csname PY@tok@sr\endcsname{\def\PY@tc##1{\textcolor[rgb]{0.73,0.40,0.53}{##1}}}
\expandafter\def\csname PY@tok@ss\endcsname{\def\PY@tc##1{\textcolor[rgb]{0.10,0.09,0.49}{##1}}}
\expandafter\def\csname PY@tok@sx\endcsname{\def\PY@tc##1{\textcolor[rgb]{0.00,0.50,0.00}{##1}}}
\expandafter\def\csname PY@tok@m\endcsname{\def\PY@tc##1{\textcolor[rgb]{0.40,0.40,0.40}{##1}}}
\expandafter\def\csname PY@tok@gh\endcsname{\let\PY@bf=\textbf\def\PY@tc##1{\textcolor[rgb]{0.00,0.00,0.50}{##1}}}
\expandafter\def\csname PY@tok@gu\endcsname{\let\PY@bf=\textbf\def\PY@tc##1{\textcolor[rgb]{0.50,0.00,0.50}{##1}}}
\expandafter\def\csname PY@tok@gd\endcsname{\def\PY@tc##1{\textcolor[rgb]{0.63,0.00,0.00}{##1}}}
\expandafter\def\csname PY@tok@gi\endcsname{\def\PY@tc##1{\textcolor[rgb]{0.00,0.63,0.00}{##1}}}
\expandafter\def\csname PY@tok@gr\endcsname{\def\PY@tc##1{\textcolor[rgb]{1.00,0.00,0.00}{##1}}}
\expandafter\def\csname PY@tok@ge\endcsname{\let\PY@it=\textit}
\expandafter\def\csname PY@tok@gs\endcsname{\let\PY@bf=\textbf}
\expandafter\def\csname PY@tok@gp\endcsname{\let\PY@bf=\textbf\def\PY@tc##1{\textcolor[rgb]{0.00,0.00,0.50}{##1}}}
\expandafter\def\csname PY@tok@go\endcsname{\def\PY@tc##1{\textcolor[rgb]{0.53,0.53,0.53}{##1}}}
\expandafter\def\csname PY@tok@gt\endcsname{\def\PY@tc##1{\textcolor[rgb]{0.00,0.27,0.87}{##1}}}
\expandafter\def\csname PY@tok@err\endcsname{\def\PY@bc##1{\setlength{\fboxsep}{0pt}\fcolorbox[rgb]{1.00,0.00,0.00}{1,1,1}{\strut ##1}}}
\expandafter\def\csname PY@tok@kc\endcsname{\let\PY@bf=\textbf\def\PY@tc##1{\textcolor[rgb]{0.00,0.50,0.00}{##1}}}
\expandafter\def\csname PY@tok@kd\endcsname{\let\PY@bf=\textbf\def\PY@tc##1{\textcolor[rgb]{0.00,0.50,0.00}{##1}}}
\expandafter\def\csname PY@tok@kn\endcsname{\let\PY@bf=\textbf\def\PY@tc##1{\textcolor[rgb]{0.00,0.50,0.00}{##1}}}
\expandafter\def\csname PY@tok@kr\endcsname{\let\PY@bf=\textbf\def\PY@tc##1{\textcolor[rgb]{0.00,0.50,0.00}{##1}}}
\expandafter\def\csname PY@tok@bp\endcsname{\def\PY@tc##1{\textcolor[rgb]{0.00,0.50,0.00}{##1}}}
\expandafter\def\csname PY@tok@fm\endcsname{\def\PY@tc##1{\textcolor[rgb]{0.00,0.00,1.00}{##1}}}
\expandafter\def\csname PY@tok@vc\endcsname{\def\PY@tc##1{\textcolor[rgb]{0.10,0.09,0.49}{##1}}}
\expandafter\def\csname PY@tok@vg\endcsname{\def\PY@tc##1{\textcolor[rgb]{0.10,0.09,0.49}{##1}}}
\expandafter\def\csname PY@tok@vi\endcsname{\def\PY@tc##1{\textcolor[rgb]{0.10,0.09,0.49}{##1}}}
\expandafter\def\csname PY@tok@vm\endcsname{\def\PY@tc##1{\textcolor[rgb]{0.10,0.09,0.49}{##1}}}
\expandafter\def\csname PY@tok@sa\endcsname{\def\PY@tc##1{\textcolor[rgb]{0.73,0.13,0.13}{##1}}}
\expandafter\def\csname PY@tok@sb\endcsname{\def\PY@tc##1{\textcolor[rgb]{0.73,0.13,0.13}{##1}}}
\expandafter\def\csname PY@tok@sc\endcsname{\def\PY@tc##1{\textcolor[rgb]{0.73,0.13,0.13}{##1}}}
\expandafter\def\csname PY@tok@dl\endcsname{\def\PY@tc##1{\textcolor[rgb]{0.73,0.13,0.13}{##1}}}
\expandafter\def\csname PY@tok@s2\endcsname{\def\PY@tc##1{\textcolor[rgb]{0.73,0.13,0.13}{##1}}}
\expandafter\def\csname PY@tok@sh\endcsname{\def\PY@tc##1{\textcolor[rgb]{0.73,0.13,0.13}{##1}}}
\expandafter\def\csname PY@tok@s1\endcsname{\def\PY@tc##1{\textcolor[rgb]{0.73,0.13,0.13}{##1}}}
\expandafter\def\csname PY@tok@mb\endcsname{\def\PY@tc##1{\textcolor[rgb]{0.40,0.40,0.40}{##1}}}
\expandafter\def\csname PY@tok@mf\endcsname{\def\PY@tc##1{\textcolor[rgb]{0.40,0.40,0.40}{##1}}}
\expandafter\def\csname PY@tok@mh\endcsname{\def\PY@tc##1{\textcolor[rgb]{0.40,0.40,0.40}{##1}}}
\expandafter\def\csname PY@tok@mi\endcsname{\def\PY@tc##1{\textcolor[rgb]{0.40,0.40,0.40}{##1}}}
\expandafter\def\csname PY@tok@il\endcsname{\def\PY@tc##1{\textcolor[rgb]{0.40,0.40,0.40}{##1}}}
\expandafter\def\csname PY@tok@mo\endcsname{\def\PY@tc##1{\textcolor[rgb]{0.40,0.40,0.40}{##1}}}
\expandafter\def\csname PY@tok@ch\endcsname{\let\PY@it=\textit\def\PY@tc##1{\textcolor[rgb]{0.25,0.50,0.50}{##1}}}
\expandafter\def\csname PY@tok@cm\endcsname{\let\PY@it=\textit\def\PY@tc##1{\textcolor[rgb]{0.25,0.50,0.50}{##1}}}
\expandafter\def\csname PY@tok@cpf\endcsname{\let\PY@it=\textit\def\PY@tc##1{\textcolor[rgb]{0.25,0.50,0.50}{##1}}}
\expandafter\def\csname PY@tok@c1\endcsname{\let\PY@it=\textit\def\PY@tc##1{\textcolor[rgb]{0.25,0.50,0.50}{##1}}}
\expandafter\def\csname PY@tok@cs\endcsname{\let\PY@it=\textit\def\PY@tc##1{\textcolor[rgb]{0.25,0.50,0.50}{##1}}}

\def\PYZbs{\char`\\}
\def\PYZus{\char`\_}
\def\PYZob{\char`\{}
\def\PYZcb{\char`\}}
\def\PYZca{\char`\^}
\def\PYZam{\char`\&}
\def\PYZlt{\char`\<}
\def\PYZgt{\char`\>}
\def\PYZsh{\char`\#}
\def\PYZpc{\char`\%}
\def\PYZdl{\char`\$}
\def\PYZhy{\char`\-}
\def\PYZsq{\char`\'}
\def\PYZdq{\char`\"}
\def\PYZti{\char`\~}
% for compatibility with earlier versions
\def\PYZat{@}
\def\PYZlb{[}
\def\PYZrb{]}
\makeatother


    % Exact colors from NB
    \definecolor{incolor}{rgb}{0.0, 0.0, 0.5}
    \definecolor{outcolor}{rgb}{0.545, 0.0, 0.0}



    
    % Prevent overflowing lines due to hard-to-break entities
    \sloppy 
    % Setup hyperref package
    \hypersetup{
      breaklinks=true,  % so long urls are correctly broken across lines
      colorlinks=true,
      urlcolor=urlcolor,
      linkcolor=linkcolor,
      citecolor=citecolor,
      }
    % Slightly bigger margins than the latex defaults
    
    \geometry{verbose,tmargin=1in,bmargin=1in,lmargin=1in,rmargin=1in}
    
    

    \begin{document}
    
    
    \maketitle
    
    

    
    \section{Opera Software - BI Intern Competence
Test}\label{opera-software---bi-intern-competence-test}

    \begin{Verbatim}[commandchars=\\\{\}]
{\color{incolor}In [{\color{incolor}1}]:} \PY{k+kn}{import} \PY{n+nn}{pandas} \PY{k}{as} \PY{n+nn}{pd}
        \PY{k+kn}{import} \PY{n+nn}{numpy} \PY{k}{as} \PY{n+nn}{np}
        \PY{k+kn}{from} \PY{n+nn}{sklearn}\PY{n+nn}{.}\PY{n+nn}{linear\PYZus{}model} \PY{k}{import} \PY{n}{LinearRegression}
        \PY{k+kn}{from} \PY{n+nn}{sklearn} \PY{k}{import} \PY{n}{metrics}
        \PY{k+kn}{import} \PY{n+nn}{statsmodels}\PY{n+nn}{.}\PY{n+nn}{formula}\PY{n+nn}{.}\PY{n+nn}{api} \PY{k}{as} \PY{n+nn}{smf}
        \PY{k+kn}{import} \PY{n+nn}{seaborn} \PY{k}{as} \PY{n+nn}{sns}
        \PY{k+kn}{import} \PY{n+nn}{matplotlib}\PY{n+nn}{.}\PY{n+nn}{pyplot} \PY{k}{as} \PY{n+nn}{plt}
        \PY{k+kn}{from} \PY{n+nn}{pandas}\PY{n+nn}{.}\PY{n+nn}{io} \PY{k}{import} \PY{n}{gbq}
\end{Verbatim}


    \subsection{Load data into database}\label{load-data-into-database}

    The provided dataset has been loaded into a database on Google BigQuery

    \begin{Verbatim}[commandchars=\\\{\}]
{\color{incolor}In [{\color{incolor}2}]:} \PY{n}{df} \PY{o}{=} \PY{n}{gbq}\PY{o}{.}\PY{n}{read\PYZus{}gbq}\PY{p}{(}\PY{l+s+s2}{\PYZdq{}}\PY{l+s+s2}{SELECT * FROM test\PYZus{}data.PRODUCT}\PY{l+s+s2}{\PYZdq{}}\PY{p}{,} \PY{n}{project\PYZus{}id}\PY{o}{=}\PY{l+s+s2}{\PYZdq{}}\PY{l+s+s2}{opera\PYZhy{}220120}\PY{l+s+s2}{\PYZdq{}}\PY{p}{)}
\end{Verbatim}


    We have now connected to the database on Google BigQuery. Let's take a
quick glance at the data we are going to work with

    \subsection{Data}\label{data}

    \begin{Verbatim}[commandchars=\\\{\}]
{\color{incolor}In [{\color{incolor}3}]:} \PY{n}{df}\PY{o}{.}\PY{n}{head}\PY{p}{(}\PY{p}{)}
\end{Verbatim}


\begin{Verbatim}[commandchars=\\\{\}]
{\color{outcolor}Out[{\color{outcolor}3}]:}   partition\_date timeperiod country     full\_version news\_locale      channel  \textbackslash{}
        0     2018-06-02 2018-06-01      HF  3.3.2254.127986       hw:en  Cross-Promo   
        1     2018-06-02 2018-06-01      BF  2.4.2254.127188       ba:fr  Cross-Promo   
        2     2018-06-02 2018-06-01      BF  3.3.2254.127986       ba:fr  Cross-Promo   
        3     2018-06-02 2018-06-01      JE  3.3.2254.127986       lv:fr  Cross-Promo   
        4     2018-06-02 2018-06-01      JE  2.4.2254.127188       lv:fr  Cross-Promo   
        
                    source  impressions  clicks  active\_users  new\_users  retained\_01  \textbackslash{}
        0  Unprecedented\_2          NaN     NaN           NaN        NaN          4.0   
        1     Widespread\_4          NaN     NaN           NaN        NaN          3.0   
        2     Widespread\_4          NaN     NaN           NaN        NaN         10.0   
        3     Widespread\_4          NaN     NaN           NaN        NaN         11.0   
        4     Widespread\_4          NaN     NaN           NaN        NaN          4.0   
        
           retained\_07  retained\_30  
        0          NaN          NaN  
        1          NaN          NaN  
        2          NaN          NaN  
        3          NaN          NaN  
        4          NaN          NaN  
\end{Verbatim}
            
    The first thing we see is that there are several null values. Let us
check how prevelent this is

    \begin{Verbatim}[commandchars=\\\{\}]
{\color{incolor}In [{\color{incolor}4}]:} \PY{n}{df}\PY{o}{.}\PY{n}{count}\PY{p}{(}\PY{p}{)}\PY{o}{/}\PY{n+nb}{len}\PY{p}{(}\PY{n}{df}\PY{p}{)}\PY{o}{*}\PY{l+m+mi}{100}
\end{Verbatim}


\begin{Verbatim}[commandchars=\\\{\}]
{\color{outcolor}Out[{\color{outcolor}4}]:} partition\_date    100.000000
        timeperiod        100.000000
        country           100.000000
        full\_version      100.000000
        news\_locale       100.000000
        channel           100.000000
        source            100.000000
        impressions        39.927218
        clicks             37.647839
        active\_users       50.300488
        new\_users          50.300488
        retained\_01         5.805593
        retained\_07         4.523071
        retained\_30         3.476683
        dtype: float64
\end{Verbatim}
            
    The table above shows the percentage of non-null data we have for each
column

    Let us find out the time range the data covers

    \begin{Verbatim}[commandchars=\\\{\}]
{\color{incolor}In [{\color{incolor}5}]:} \PY{n+nb}{max}\PY{p}{(}\PY{n}{df}\PY{p}{[}\PY{l+s+s2}{\PYZdq{}}\PY{l+s+s2}{partition\PYZus{}date}\PY{l+s+s2}{\PYZdq{}}\PY{p}{]}\PY{p}{)}
\end{Verbatim}


\begin{Verbatim}[commandchars=\\\{\}]
{\color{outcolor}Out[{\color{outcolor}5}]:} Timestamp('2018-08-30 00:00:00')
\end{Verbatim}
            
    \begin{Verbatim}[commandchars=\\\{\}]
{\color{incolor}In [{\color{incolor}6}]:} \PY{n+nb}{min}\PY{p}{(}\PY{n}{df}\PY{p}{[}\PY{l+s+s2}{\PYZdq{}}\PY{l+s+s2}{partition\PYZus{}date}\PY{l+s+s2}{\PYZdq{}}\PY{p}{]}\PY{p}{)}
\end{Verbatim}


\begin{Verbatim}[commandchars=\\\{\}]
{\color{outcolor}Out[{\color{outcolor}6}]:} Timestamp('2018-06-01 00:00:00')
\end{Verbatim}
            
    The data covers three months. From first of June to 30th of August 2018.

    \subsection{Average daily new user
run-rate}\label{average-daily-new-user-run-rate}

    \begin{Verbatim}[commandchars=\\\{\}]
{\color{incolor}In [{\color{incolor}7}]:} \PY{n}{new\PYZus{}user\PYZus{}run\PYZus{}rate} \PY{o}{=} \PY{n}{df}\PY{o}{.}\PY{n}{groupby}\PY{p}{(}\PY{p}{[}\PY{l+s+s2}{\PYZdq{}}\PY{l+s+s2}{country}\PY{l+s+s2}{\PYZdq{}}\PY{p}{,} \PY{l+s+s2}{\PYZdq{}}\PY{l+s+s2}{timeperiod}\PY{l+s+s2}{\PYZdq{}}\PY{p}{]}\PY{p}{)}\PY{o}{.}\PY{n}{sum}\PY{p}{(}\PY{p}{)}\PY{o}{.}\PY{n}{groupby}\PY{p}{(}\PY{l+s+s2}{\PYZdq{}}\PY{l+s+s2}{country}\PY{l+s+s2}{\PYZdq{}}\PY{p}{)}\PY{p}{[}\PY{l+s+s2}{\PYZdq{}}\PY{l+s+s2}{new\PYZus{}users}\PY{l+s+s2}{\PYZdq{}}\PY{p}{]}\PY{o}{.}\PY{n}{mean}\PY{p}{(}\PY{p}{)}
        \PY{n}{new\PYZus{}user\PYZus{}run\PYZus{}rate}
\end{Verbatim}


\begin{Verbatim}[commandchars=\\\{\}]
{\color{outcolor}Out[{\color{outcolor}7}]:} country
        BF    1225.262295
        DC    2870.360656
        GT    2201.688525
        HF    2455.672131
        IB    1568.590164
        JE    1613.098361
        JY     661.688525
        KI    1804.049180
        LP     169.622951
        LQ     830.213115
        LY     144.213115
        OU     355.491803
        RE    2448.491803
        RO     399.754098
        TT     763.262295
        TZ     302.819672
        UF    2678.606557
        WA     226.065574
        XJ    1225.213115
        ZY    2368.557377
        Name: new\_users, dtype: float64
\end{Verbatim}
            
    This is the average daily new users for the three month period

    \subsection{Top 10 sources by new users in
June}\label{top-10-sources-by-new-users-in-june}

    \begin{Verbatim}[commandchars=\\\{\}]
{\color{incolor}In [{\color{incolor}8}]:} \PY{n}{june} \PY{o}{=} \PY{n}{df}\PY{p}{[}\PY{n}{pd}\PY{o}{.}\PY{n}{to\PYZus{}datetime}\PY{p}{(}\PY{n}{df}\PY{p}{[}\PY{l+s+s2}{\PYZdq{}}\PY{l+s+s2}{timeperiod}\PY{l+s+s2}{\PYZdq{}}\PY{p}{]}\PY{p}{)}\PY{o}{.}\PY{n}{dt}\PY{o}{.}\PY{n}{month} \PY{o}{==} \PY{l+m+mi}{6}\PY{p}{]}
        \PY{n}{top\PYZus{}ten\PYZus{}sources\PYZus{}june} \PY{o}{=} \PY{n}{june}\PY{o}{.}\PY{n}{groupby}\PY{p}{(}\PY{p}{[}\PY{l+s+s2}{\PYZdq{}}\PY{l+s+s2}{source}\PY{l+s+s2}{\PYZdq{}}\PY{p}{]}\PY{p}{)}\PY{o}{.}\PY{n}{sum}\PY{p}{(}\PY{p}{)}\PY{o}{.}\PY{n}{nlargest}\PY{p}{(}\PY{l+m+mi}{10}\PY{p}{,} \PY{l+s+s2}{\PYZdq{}}\PY{l+s+s2}{new\PYZus{}users}\PY{l+s+s2}{\PYZdq{}}\PY{p}{)}
\end{Verbatim}


    \begin{Verbatim}[commandchars=\\\{\}]
{\color{incolor}In [{\color{incolor}9}]:} \PY{n}{july} \PY{o}{=} \PY{n}{df}\PY{p}{[}\PY{n}{pd}\PY{o}{.}\PY{n}{to\PYZus{}datetime}\PY{p}{(}\PY{n}{df}\PY{p}{[}\PY{l+s+s2}{\PYZdq{}}\PY{l+s+s2}{timeperiod}\PY{l+s+s2}{\PYZdq{}}\PY{p}{]}\PY{p}{)}\PY{o}{.}\PY{n}{dt}\PY{o}{.}\PY{n}{month} \PY{o}{==} \PY{l+m+mi}{7}\PY{p}{]}
\end{Verbatim}


    \begin{Verbatim}[commandchars=\\\{\}]
{\color{incolor}In [{\color{incolor}10}]:} \PY{n}{df\PYZus{}top\PYZus{}ten\PYZus{}july} \PY{o}{=} \PY{n}{july}\PY{o}{.}\PY{n}{loc}\PY{p}{[}\PY{n}{july}\PY{p}{[}\PY{l+s+s2}{\PYZdq{}}\PY{l+s+s2}{source}\PY{l+s+s2}{\PYZdq{}}\PY{p}{]}\PY{o}{.}\PY{n}{isin}\PY{p}{(}\PY{n+nb}{list}\PY{p}{(}\PY{n}{top\PYZus{}ten\PYZus{}sources\PYZus{}june}\PY{o}{.}\PY{n}{index}\PY{o}{.}\PY{n}{values}\PY{p}{)}\PY{p}{)}\PY{p}{]}
\end{Verbatim}


    \begin{Verbatim}[commandchars=\\\{\}]
{\color{incolor}In [{\color{incolor}11}]:} \PY{n}{df\PYZus{}top\PYZus{}ten\PYZus{}july}\PY{o}{.}\PY{n}{groupby}\PY{p}{(}\PY{p}{[}\PY{l+s+s2}{\PYZdq{}}\PY{l+s+s2}{source}\PY{l+s+s2}{\PYZdq{}}\PY{p}{,} \PY{l+s+s2}{\PYZdq{}}\PY{l+s+s2}{timeperiod}\PY{l+s+s2}{\PYZdq{}}\PY{p}{]}\PY{p}{)}\PY{o}{.}\PY{n}{sum}\PY{p}{(}\PY{p}{)}\PY{o}{.}\PY{n}{groupby}\PY{p}{(}\PY{l+s+s2}{\PYZdq{}}\PY{l+s+s2}{source}\PY{l+s+s2}{\PYZdq{}}\PY{p}{)}\PY{p}{[}\PY{l+s+s2}{\PYZdq{}}\PY{l+s+s2}{active\PYZus{}users}\PY{l+s+s2}{\PYZdq{}}\PY{p}{]}\PY{o}{.}\PY{n}{mean}\PY{p}{(}\PY{p}{)}\PY{o}{.}\PY{n}{sort\PYZus{}values}\PY{p}{(}\PY{n}{ascending}\PY{o}{=}\PY{k+kc}{False}\PY{p}{)}
\end{Verbatim}


\begin{Verbatim}[commandchars=\\\{\}]
{\color{outcolor}Out[{\color{outcolor}11}]:} source
         Cryptic\_4       84564.870968
         Indeed\_1        48873.000000
         Professor\_7     47671.741935
         Latent\_4        44108.903226
         Febrile\_8       32720.161290
         Widespread\_4    10841.838710
         Dreams\_4        10091.935484
         Learned\_3        8253.225806
         Family\_1         7716.000000
         Returned\_5       7142.129032
         Name: active\_users, dtype: float64
\end{Verbatim}
            
    This is the average active user in July for the top ten sources by user
count in June

    \subsection{Top 5 countries by average retention ratio in
June}\label{top-5-countries-by-average-retention-ratio-in-june}

    \begin{Verbatim}[commandchars=\\\{\}]
{\color{incolor}In [{\color{incolor}12}]:} \PY{n}{june\PYZus{}mean} \PY{o}{=} \PY{n}{june}\PY{o}{.}\PY{n}{groupby}\PY{p}{(}\PY{p}{[}\PY{l+s+s2}{\PYZdq{}}\PY{l+s+s2}{country}\PY{l+s+s2}{\PYZdq{}}\PY{p}{,} \PY{l+s+s2}{\PYZdq{}}\PY{l+s+s2}{timeperiod}\PY{l+s+s2}{\PYZdq{}}\PY{p}{]}\PY{p}{)}\PY{o}{.}\PY{n}{sum}\PY{p}{(}\PY{p}{)}\PY{o}{.}\PY{n}{groupby}\PY{p}{(}\PY{l+s+s2}{\PYZdq{}}\PY{l+s+s2}{country}\PY{l+s+s2}{\PYZdq{}}\PY{p}{)}\PY{p}{[}\PY{l+s+s2}{\PYZdq{}}\PY{l+s+s2}{new\PYZus{}users}\PY{l+s+s2}{\PYZdq{}}\PY{p}{,} \PY{l+s+s2}{\PYZdq{}}\PY{l+s+s2}{retained\PYZus{}30}\PY{l+s+s2}{\PYZdq{}}\PY{p}{]}\PY{o}{.}\PY{n}{mean}\PY{p}{(}\PY{p}{)}
\end{Verbatim}


    \begin{Verbatim}[commandchars=\\\{\}]
{\color{incolor}In [{\color{incolor}13}]:} \PY{n}{june\PYZus{}mean}\PY{p}{[}\PY{l+s+s2}{\PYZdq{}}\PY{l+s+s2}{r30\PYZus{}ratio}\PY{l+s+s2}{\PYZdq{}}\PY{p}{]} \PY{o}{=} \PY{n}{june\PYZus{}mean}\PY{p}{[}\PY{l+s+s2}{\PYZdq{}}\PY{l+s+s2}{retained\PYZus{}30}\PY{l+s+s2}{\PYZdq{}}\PY{p}{]} \PY{o}{/} \PY{n}{june\PYZus{}mean}\PY{p}{[}\PY{l+s+s2}{\PYZdq{}}\PY{l+s+s2}{new\PYZus{}users}\PY{l+s+s2}{\PYZdq{}}\PY{p}{]}
\end{Verbatim}


    \begin{Verbatim}[commandchars=\\\{\}]
{\color{incolor}In [{\color{incolor}14}]:} \PY{n}{top\PYZus{}five\PYZus{}ratio\PYZus{}june} \PY{o}{=} \PY{n}{june\PYZus{}mean}\PY{o}{.}\PY{n}{nlargest}\PY{p}{(}\PY{l+m+mi}{5}\PY{p}{,} \PY{l+s+s2}{\PYZdq{}}\PY{l+s+s2}{r30\PYZus{}ratio}\PY{l+s+s2}{\PYZdq{}}\PY{p}{)}
         \PY{n}{top\PYZus{}five\PYZus{}ratio\PYZus{}june}
\end{Verbatim}


\begin{Verbatim}[commandchars=\\\{\}]
{\color{outcolor}Out[{\color{outcolor}14}]:}            new\_users  retained\_30  r30\_ratio
         country                                     
         OU        330.333333    71.766667   0.217255
         KI       1789.133333   350.433333   0.195868
         JY        621.366667   112.700000   0.181374
         HF       2470.000000   441.133333   0.178596
         DC       2700.000000   445.966667   0.165173
\end{Verbatim}
            
    These are the top five countries for the daily R30 retention ratio

    \subsection{User aquisition trends}\label{user-aquisition-trends}

    In the following we assume that the ratio of users retained after 30
days reflect the true growth of good quality users. We will also assume
that the ratio between clicks and impressions (Click-Through-Rate or
CTR) is a good measure for the effectiveness of a source

    \begin{Verbatim}[commandchars=\\\{\}]
{\color{incolor}In [{\color{incolor}15}]:} \PY{n}{sources} \PY{o}{=} \PY{n}{df}\PY{o}{.}\PY{n}{groupby}\PY{p}{(}\PY{p}{[}\PY{l+s+s2}{\PYZdq{}}\PY{l+s+s2}{source}\PY{l+s+s2}{\PYZdq{}}\PY{p}{,} \PY{l+s+s2}{\PYZdq{}}\PY{l+s+s2}{timeperiod}\PY{l+s+s2}{\PYZdq{}}\PY{p}{]}\PY{p}{)}\PY{o}{.}\PY{n}{sum}\PY{p}{(}\PY{p}{)}\PY{o}{.}\PY{n}{groupby}\PY{p}{(}\PY{l+s+s2}{\PYZdq{}}\PY{l+s+s2}{source}\PY{l+s+s2}{\PYZdq{}}\PY{p}{)}\PY{o}{.}\PY{n}{sum}\PY{p}{(}\PY{p}{)}
         \PY{n}{sources}\PY{p}{[}\PY{l+s+s2}{\PYZdq{}}\PY{l+s+s2}{growth}\PY{l+s+s2}{\PYZdq{}}\PY{p}{]} \PY{o}{=} \PY{n}{sources}\PY{p}{[}\PY{l+s+s2}{\PYZdq{}}\PY{l+s+s2}{retained\PYZus{}30}\PY{l+s+s2}{\PYZdq{}}\PY{p}{]} \PY{o}{/} \PY{n}{sources}\PY{p}{[}\PY{l+s+s2}{\PYZdq{}}\PY{l+s+s2}{new\PYZus{}users}\PY{l+s+s2}{\PYZdq{}}\PY{p}{]}
         \PY{n}{sources}\PY{p}{[}\PY{l+s+s2}{\PYZdq{}}\PY{l+s+s2}{CTR}\PY{l+s+s2}{\PYZdq{}}\PY{p}{]} \PY{o}{=} \PY{n}{sources}\PY{p}{[}\PY{l+s+s2}{\PYZdq{}}\PY{l+s+s2}{clicks}\PY{l+s+s2}{\PYZdq{}}\PY{p}{]} \PY{o}{/} \PY{n}{sources}\PY{p}{[}\PY{l+s+s2}{\PYZdq{}}\PY{l+s+s2}{impressions}\PY{l+s+s2}{\PYZdq{}}\PY{p}{]}
\end{Verbatim}


    \begin{Verbatim}[commandchars=\\\{\}]
{\color{incolor}In [{\color{incolor}16}]:} \PY{n}{sns}\PY{o}{.}\PY{n}{pairplot}\PY{p}{(}\PY{n}{sources}\PY{p}{,} \PY{n}{x\PYZus{}vars}\PY{o}{=}\PY{p}{[}\PY{l+s+s2}{\PYZdq{}}\PY{l+s+s2}{CTR}\PY{l+s+s2}{\PYZdq{}}\PY{p}{]}\PY{p}{,} \PY{n}{y\PYZus{}vars}\PY{o}{=}\PY{l+s+s2}{\PYZdq{}}\PY{l+s+s2}{growth}\PY{l+s+s2}{\PYZdq{}}\PY{p}{,} \PY{n}{size}\PY{o}{=}\PY{l+m+mi}{10}\PY{p}{,} \PY{n}{aspect}\PY{o}{=}\PY{l+m+mf}{0.7}\PY{p}{,} \PY{n}{kind}\PY{o}{=}\PY{l+s+s2}{\PYZdq{}}\PY{l+s+s2}{reg}\PY{l+s+s2}{\PYZdq{}}\PY{p}{)}
\end{Verbatim}


\begin{Verbatim}[commandchars=\\\{\}]
{\color{outcolor}Out[{\color{outcolor}16}]:} <seaborn.axisgrid.PairGrid at 0x7f83e8616ba8>
\end{Verbatim}
            
    \begin{center}
    \adjustimage{max size={0.9\linewidth}{0.9\paperheight}}{output_32_1.png}
    \end{center}
    { \hspace*{\fill} \\}
    
    From the scatterplot we can se that there is seems to be a strong
relationship between CTR and growth as we have defined them

    \begin{Verbatim}[commandchars=\\\{\}]
{\color{incolor}In [{\color{incolor}17}]:} \PY{n}{feature\PYZus{}cols} \PY{o}{=} \PY{p}{[}\PY{l+s+s2}{\PYZdq{}}\PY{l+s+s2}{CTR}\PY{l+s+s2}{\PYZdq{}}\PY{p}{,} \PY{l+s+s2}{\PYZdq{}}\PY{l+s+s2}{clicks}\PY{l+s+s2}{\PYZdq{}}\PY{p}{,} \PY{l+s+s2}{\PYZdq{}}\PY{l+s+s2}{impressions}\PY{l+s+s2}{\PYZdq{}}\PY{p}{]}
         \PY{n}{X} \PY{o}{=} \PY{n}{sources}\PY{p}{[}\PY{n}{feature\PYZus{}cols}\PY{p}{]}
         \PY{n}{y} \PY{o}{=} \PY{n}{sources}\PY{o}{.}\PY{n}{growth}
         
         \PY{n}{linreg} \PY{o}{=} \PY{n}{LinearRegression}\PY{p}{(}\PY{p}{)}
         \PY{n}{linreg}\PY{o}{.}\PY{n}{fit}\PY{p}{(}\PY{n}{X}\PY{p}{,} \PY{n}{y}\PY{p}{)}
         
         \PY{k}{for} \PY{n}{i} \PY{o+ow}{in} \PY{n+nb}{zip}\PY{p}{(}\PY{n}{feature\PYZus{}cols}\PY{p}{,} \PY{n}{linreg}\PY{o}{.}\PY{n}{coef\PYZus{}}\PY{p}{)}\PY{p}{:}
             \PY{n+nb}{print}\PY{p}{(}\PY{n}{i}\PY{p}{)}
\end{Verbatim}


    \begin{Verbatim}[commandchars=\\\{\}]
('CTR', 5.018666746088409)
('clicks', -2.7433881657443655e-08)
('impressions', 2.0739441029407954e-09)

    \end{Verbatim}

    We see that one unit increase in CTR will on average yield 5 unit
increases in growth

    \begin{Verbatim}[commandchars=\\\{\}]
{\color{incolor}In [{\color{incolor}18}]:} \PY{n}{lm} \PY{o}{=} \PY{n}{smf}\PY{o}{.}\PY{n}{ols}\PY{p}{(}\PY{n}{formula}\PY{o}{=}\PY{l+s+s1}{\PYZsq{}}\PY{l+s+s1}{growth \PYZti{} CTR}\PY{l+s+s1}{\PYZsq{}}\PY{p}{,} \PY{n}{data}\PY{o}{=}\PY{n}{sources}\PY{p}{)}\PY{o}{.}\PY{n}{fit}\PY{p}{(}\PY{p}{)}
         \PY{n+nb}{print}\PY{p}{(}\PY{n}{lm}\PY{o}{.}\PY{n}{pvalues}\PY{p}{)}
\end{Verbatim}


    \begin{Verbatim}[commandchars=\\\{\}]
Intercept    0.185124
CTR          0.015290
dtype: float64

    \end{Verbatim}

    This means that we can reject the null hypothesis and declare that there
is a relationship between CTR and growth

    Let us now create a combination of CTR and growth as a score feature and
determine the top 5 sources based upon that feature

    \begin{Verbatim}[commandchars=\\\{\}]
{\color{incolor}In [{\color{incolor}19}]:} \PY{n}{sources}\PY{p}{[}\PY{l+s+s2}{\PYZdq{}}\PY{l+s+s2}{score}\PY{l+s+s2}{\PYZdq{}}\PY{p}{]} \PY{o}{=} \PY{n}{sources}\PY{p}{[}\PY{l+s+s2}{\PYZdq{}}\PY{l+s+s2}{CTR}\PY{l+s+s2}{\PYZdq{}}\PY{p}{]} \PY{o}{*} \PY{n}{sources}\PY{p}{[}\PY{l+s+s2}{\PYZdq{}}\PY{l+s+s2}{growth}\PY{l+s+s2}{\PYZdq{}}\PY{p}{]}
         \PY{n}{sources}\PY{o}{.}\PY{n}{nlargest}\PY{p}{(}\PY{l+m+mi}{5}\PY{p}{,} \PY{l+s+s2}{\PYZdq{}}\PY{l+s+s2}{score}\PY{l+s+s2}{\PYZdq{}}\PY{p}{)}
\end{Verbatim}


\begin{Verbatim}[commandchars=\\\{\}]
{\color{outcolor}Out[{\color{outcolor}19}]:}                  impressions     clicks  active\_users  new\_users  retained\_01  \textbackslash{}
         source                                                                          
         Unprecedented\_2    1647563.0   158255.0       16079.0      612.0        376.0   
         Astonished\_8      17061670.0  1493145.0      187238.0     7508.0       4488.0   
         Interview\_5        1226255.0    95722.0        8174.0      385.0        284.0   
         Professor\_3        3826121.0   343174.0       40316.0     1160.0        618.0   
         Mentioned\_4        1215555.0    97399.0       13759.0      800.0        421.0   
         
                          retained\_07  retained\_30    growth       CTR     score  
         source                                                                   
         Unprecedented\_2        253.0        179.0  0.292484  0.096054  0.028094  
         Astonished\_8          3058.0       2271.0  0.302477  0.087515  0.026471  
         Interview\_5            188.0        123.0  0.319481  0.078060  0.024939  
         Professor\_3            450.0        288.0  0.248276  0.089692  0.022268  
         Mentioned\_4            211.0        156.0  0.195000  0.080127  0.015625  
\end{Verbatim}
            
    \subsubsection{Conclusion}\label{conclusion}

Based upon our previous stated assumptions and the assumption that we
can use a linear model to determine a relationship between our features,
we have found that the table will show the sources that will bring good
users in volume


    % Add a bibliography block to the postdoc
    
    
    
    \end{document}
